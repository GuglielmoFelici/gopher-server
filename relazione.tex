\documentclass{article}

\usepackage{hyperref}
\usepackage{listings}
\usepackage{color}

\definecolor{dkgreen}{rgb}{0,0.6,0}
\definecolor{gray}{rgb}{0.5,0.5,0.5}
\definecolor{mauve}{rgb}{0.58,0,0.82}

\lstset{frame=tb,
  language=C,
  aboveskip=3mm,
  belowskip=3mm,
  showstringspaces=false,
  columns=flexible,
  basicstyle={\small\ttfamily},
  numbers=none,
  numberstyle=\tiny\color{gray},
  keywordstyle=\color{blue},
  commentstyle=\color{dkgreen},
  stringstyle=\color{mauve},
  breaklines=true,
  breakatwhitespace=true,
  tabsize=3
}

\begin{document}

\tableofcontents
%newpage

\section{Utilizzo del software}

\newpage

\section{Struttura del progetto}

Il progetto è strutturato con un approccio modulare gerarchico. 
Ogni modulo esegue uno specifico compito all'interno dell'applicazione, servendosi delle funzionalità
dei livelli inferiori ed esponendo funzionalità ai livelli superiori.
A livello implementativo, a ogni modulo corrisponde uno o più header e uno o più file sorgente.
I quattro moduli sono:
\begin{itemize}
    \item Server - Riceve le richieste di connessione dei client e crea nuovi thread 
    o processi per inviare le risposte. Si occupa inoltre della lettura dei file di configurazione e 
    della gestione dei segnali.
    \item Protocollo - Legge la richieste di un client e costruisce la risposta secondo lo standard del
    protocollo Gopher. In questo modulo avviene la lettura del file system e l'eventuale mapping 
    in memoria dei file.
    \item Logger - Gestisce il processo di logging, che ha il compito di registrare le operazioni di invio
    di file.
    \item Portabilità - API minimale che offre diverse funzioni di supporto portabili. Espone wrapper
    di varie chiamate di sistema e altre operazioni comuni non portabili, nonché wrapper di tipi di dato.
\end{itemize} 
Di seguito vengono esaminati in astratto i moduli nel dettaglio.

\subsection{Server}
Il modulo Server si occupa di leggere il file di configurazione, di inizializzare la Winsock DLL 
(caso Windows), di amministrare le connessioni tramite socket e di gestire i 
segnali per l'interazione con l'utente.
Quando riceve una richiesta di connessione, il modulo si occupa di generare un nuovo thread o un nuovo processo
e passare il socket connesso al modulo Protocollo.
Il modulo Server espone un'interfaccia composta da una struttura dati (server\_t) e
varie funzioni per inizializzare il socket, leggere le configurazioni, installare i 
gestori di segnali di default e mettere in ascolto il server.  

\subsection{Protocollo}
Il modulo Protocollo riceve in input un socket connesso.
Esso si occupa di leggere, validare, interpretare e soddisfare la richiesta del client.
Nel caso in cui venga inviato un file, il modulo Protocollo richiede al modulo Logger di registrare 
l'avvenuto trasferimento.

\subsection{Logger}
Il modulo Logger si interfaccia con i moduli superiori attraverso una struttura dati (logger\_t) che
contiene le informazioni sul processo di logging, e le funzioni per avviare o interrompere il processo
e loggare un trasferimento.
E' implementato dall'header logger.h e dal file logger.c, nonché dal file winLogger.c nel caso di Windows. 


\subsection{Portabilità}
Il modulo portabilità offre numerose funzionalità implementate in modo portabile per essere riutilizzate
dai moduli che lo richiedano. Le funzioni si articolano in:
\begin{itemize}
    \item utilità generiche (operazioni sulla working directory, logging...)
    \item operazioni sui socket
    \item operazioni su thread e processi
    \item operazioni sul file system
    \item wrapper per tipi di dato
    \item messaggi di log e debug uniformi
    \item parsing di opzioni da riga di comando.
\end{itemize}
Il modulo Portabilità è implementato dagli header platform.h, datatypes.h, log.h e wingetopt.h, e dai 
sorgenti platform.c e \href{http://note.sonots.com/Comp/CompLang/cpp/getopt.html}{wingetopt.c}. 
\newpage


\section{Scelte implementative}

Per quanto il progetto sia stato impostato con una struttura modulare per trarre tutti i vantaggi della
separation of concerns in fase di sviluppo, le sue componenti sono comunque interconnesse e progettate per coesistere 
in un'unica applicazione. Da qui la scelta di utilizzare configurazioni globali 
e usare path relativi che assumono la struttura di file e directory.\\

Per realizzare l'aspetto modulare ogni file espone, tramite il proprio header, funzioni utili a espletare
i compiti previsti dal modulo logico a cui il file fa riferimento. Le funzioni di ogni modulo
utilizzano come valore di ritorno delle costanti per indicare il successo o il fallimento dell'operazione (a volte
messe in OR con altre costanti per fornire ulteriori informazioni sull'esito).
\\Le funzioni esposte utilizzano i tipi definiti in datatypes.h, per astrarre dai tipi specifici utilizzati
dalla libreria Win32 o dalle funzioni POSIX. La realizzazione interna delle funzioni invece
 utilizza i tipi specifici della piattaforma, e consiste, laddove impossibile scrivere codice portabile o utilizzare solo funzioni offerte da Portabilità,
 di un'implementazione ad hoc per Windows e una per GNU/Linux, demandando la scelta dell'implementazione
da utilizzare al preprocessore.
Una documentazione approfondita su ogni file, funzione e struttura, generata con Doxygen a partire dalla 
documentazione sul codice sorgente, è disponibile \href{html/index.html}{qui}. 

\subsection{Server}
Il modulo Server è implementato dall'header server.h, dal file server.c e dal file winGopherProcess.c 
nel caso in cui si compili su Windows.\\
L'header dichiara la struttura server\_t, utilizzata come interfaccia per il modulo:
\begin{lstlisting}
/** A struct representing an instance of a gopher server */
typedef struct {
    /** The socket of the server */
    socket_t sock;
    /** The socket address */
    struct sockaddr_in sockAddr;
    /** The port where the server is listening */
    unsigned short port;
    /** A flag indicating whether the server should spawn a process or a thread per request */
    bool multiProcess;
} server_t;
\end{lstlisting}
La maggior parte delle funzioni dichiarate dall'header richiede in input
un puntatore a una struttura server\_t, dalla quale leggere o sulla quale scrivere informazioni.

\subsubsection{Lettura della configurazione}

La lettura dei file di configurazione avviene tramite la funzione \href{}
{readConfig}, che legge, mediante le funzioni della libreria standard, il file identificato 
dal path \href{}{CONFIG\_FILE} e scrive le configurazioni nella struttura server\_t ricevuta in input.
Il file è una sequenze di righe del tipo CHIAVE=VALORE.
La lettura avviene all'avvio dell'applicazione e ogniqualvolta venga ricevuto il segnale 
SIGHUP (caso Linux) o CTRL\_BREAK (caso Windows) durante il \href{}
{main loop} del server.\\
I gestori dei segnali utilizzano le variabili booleane globali
\begin{lstlisting}
    static sig_atomic volatile updateConfig = false;
    static sig_atomic volatile requestShutdown = false;
\end{lstlisting}
che vengono lette periodicamente fintanto che il server è in ascolto. L'accesso a queste variabili è 
protetto da una CRITICAL\_SECTION nel caso Windows e dall'uso del tipo sig\_atomic\_t (wrappato in sig\_atomic per comodità)
nel caso Linux, che assicura l'accesso atomico alle variabili anche in presenza di interrupt asincroni.
Nel caso in cui la lettura delle configurazioni fallisse vengono utilizzate le 
impostazioni di default, salvate in globals.h.

\subsubsection{Avvio del server}

Il server viene avviato con la funzione \href{}{runServer},
che prende in input un puntatore a una struttura server\_t inizializzata con le funzioni precedenti.
La funzione attende con una select una richiesta di connessione, controllando periodicamente la ricezione
di segnali. Questa scelta è motivata dal fatto che gli eventi da console di Windows non interrompono la select.
Quando il server riceve una richiesta di connessione, genera un nuovo thread o processo (a seconda della modalità
utilizzata), il quale passa il controllo al protocollo tramite la funzione \href{}
{gopher}. Siccome il modulo Protocollo si assume la responsabilità di avviare un nuovo thread se richiesto
l'invio di un file, il thread principale di un processo che invoca la funzione gopher non termina,
se non tramite le funzioni pthread\_exit o ExitThread. Quando il server riceve una richiesta di terminazione, 
la funzione runServer termina con successo. Il modulo Server non si occupa di liberare le risorse.

\subsection{Protocollo}

Il modulo Protocollo è implementato dall'header protocol.h e dal file protocol.c.
L'header espone all'esterno, oltre a diverse costanti, una singola funzione \href{}{gopher}.
Il modulo Protocollo si occupa di chiudere il socket del client sia in caso di fallimento che in caso di errore.
La funzione riceve l'input dal socket ricevuto fin quando non legge un CRLF, come da protocollo.
Se la connessione viene chiusa o si verifica un errore, la funzione fallisce.
L'input letto viene validato (poiché il selettore viene utilizzato per recuperare la risorsa
richiesta, non può contenere costrutti per la navigazione del file system, quali ./ ../ etc) e normalizzato,
per consentire una maggiore flessibilità circa le richieste accettate.

\subsubsection{Costruzione della risposta}
La lista di file disponibili, se richiesta una directory, viene costruita utilizzando la funzione \href{}
{iterateDir}, la quale funge da interfaccia per le primitive specifiche delle piattaforme Linux e Windows,
operando in modo analogo.\\
Il carattere prefisso a ogni riga della risposta, che rappresenta il tipo del file, viene calcolato
cercando determinate parole chiave nell'output del comando "file" nel caso Linux, e leggendo invece
l'estensione nel caso Windows. Le estensioni riconosciute sono memorizzate nell'array \href{}{extensions}.\\
Se viene richiesto un file, questo viene mappato in memoria utilizzando la funzione di 
Portabilità \href{}{getFileMap}, che wrappa le rispettive funzioni
mmap di Linux e CreateFileMapping/MapViewOfFile di Windows. Viene poi creato un nuovo thread, che si occupa di
inviare il file mappato in memoria al client, di chiudere la connessione e di costruire una stringa da passare al
modulo Logger tramite la funzione \href{}{logTransfer}

\subsection{Logger}
Il modulo Logger è implementato dall'header logger.h, dal file logger.c e in aggiunta dal file winLogger.c 
nel caso in cui si compili su Windows.\\
L'header dichiara la struttura logger\_t, utilizzata come interfaccia per il modulo:
\begin{lstlisting}
    /**
    * A struct representing an instance of a transfer logger
   */
   typedef struct {
       /** Pipe to use for read/write */
       pipe_t logPipe;
       /** Pointer to the mutex guarding the log pipe */
       mutex_t* pLogMutex;
       /** [Linux only] Pointer to the condition variable to notify the logger for incoming data */
       cond_t* pLogCond;
       /** [Windows only] Event to notify the logger for incoming data */
       event_t logEvent;
       /** The pid of the log process */
       proc_id_t pid;
   } logger_t;
\end{lstlisting}
Come nel caso del server, le funzioni comunicano con i rispettivi chiamanti tramite puntatori a 
strutture logger_t.
L'header espone tre funzioni: per avviare il logger, per interromperlo e per loggare una stringa.
A causa delle differenze tra Windows e Linux circa l'IPC, le due implementazioni sono completamente distinte.
\subsubsection{Windows}



\end{document}