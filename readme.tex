 
\documentclass{article}

\usepackage{hyperref}

\usepackage{listings}

\lstset{frame=tb,
  language=C,
  aboveskip=3mm,
  belowskip=3mm,
  showstringspaces=false,
  columns=flexible,
  basicstyle={\small\ttfamily},
  numbers=none,
  numberstyle=\tiny\color{gray},
  keywordstyle=\color{blue},
  commentstyle=\color{dkgreen},
  stringstyle=\color{mauve},
  breakatwhitespace=true,
  tabsize=3
}

\begin{document}

\section*{Gopher Server}

\subsection*{Utilizzo del software}

E' possibile compilare il progetto tramite make. La compilazione genera l'eseguibile linuxserver o,
nel caso in cui si esegua la build su Windows, il file winserver.exe e due binari aggiuntivi.
Viene effettuato il linking alla libreria pthread su GNU/Linux e alla libreria Winsock (Ws2\_32) su Windows.\\
Il programma riconosce le seguenti opzioni da riga di comando:\\
\begin{lstlisting}
-d DIR
    Utilizza la directory DIR come root del server.
-f FILE 
    Leggi le configurazioni da FILE. 
    Per informazioni sulla struttura del file, vedere la sezione 
    1.1 Lettura della configurazione.
-h
    Mostra la sintassi.
-l FILE
    Logga i trasferimenti sul file FILE. 
    Default: logfile.
-m 
    Utilizza un processo per ogni client. 
    Default: no.
-p PORT
    Metti il server in ascolto sulla porta PORT. 
    Default: 7070.
-v
    Imposta il livello di log informativi del server 
    (syslog su Linux, console su Windows). I livelli sono:
    0 - Non loggare nulla.
    1 - Logga gli errori.
    2 - Logga gli errori e i warning.
    3 - Logga gli errori, i warning e i messaggi di info.
    4 - Logga errori, warning, messaggi di info e messaggi di debug.
    Default: 2.

\end{lstlisting}
Le opzioni da riga di comando hanno precedenza su quelle scritte nel file di configurazione.\\
Se non viene specificata alcuna root, verrà usata la directory contenente l'eseguibile.

\subsection*{Lettura della configurazione}

Se viene utilizzato un file di configurazione con l'opzione -f, le impostazioni non specificate
da riga di comando vengono lette dal file dato in input. Questo deve essere una sequenza di righe del tipo CHIAVE = VALORE, 
dove sono accettate le seguenti coppie:
\begin{itemize}
    \item port = $<$numero di porta$>$
    \item multiprocess = yes/no
    \item verbosity = {1..4}
    \item logfile = $<$path$>$
    \item root = $<$path$>$
\end{itemize}
La lettura avviene all'avvio dell'applicazione e ogniqualvolta venga ricevuto il segnale 
SIGHUP (caso Linux) o CTRL\_BREAK (caso Windows) durante il \href{html/server_8h.html#a533c9a4292e9d1106ff7c54fbf75090a}
{main loop} del server; tuttavia, nel caso in cui si ricarichino le configurazioni mentre il server è in esecuzione,
non verranno aggiornati né il path di logging né la root directory.\\

\end{document}
